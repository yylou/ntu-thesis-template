\begin{Abstract_E}

Wearable devices have become ubiquitous and provide convenient features for users, 
such as heart rate monitor, fitness tracker.
However, because the devices are designed to be lightweight and power-saving,
the computing capability of wearable devices is often limited. 
The existing solution is to pair with the local-hub, mostly smartphone, via Bluetooth Low Energy (BLE) or Wi-Fi interface.
Whenever a request takes place, 
the wearable device will automatically transfer it to local-hub and then reply with the results.
Connection via bluetooth, however, is only functional within a narrow connection range, 
whereas connection via Wi-Fi interface has a long response time. 
It means when you left your smartphone at home and walk out with your wearable device only, 
you need to compromise on long network latency delay. 
In this paper, we propose a system based on the concept of Fog Computing to address those shortcomings. 
We deploy the wearable services at the edge side to serve the wearable devices.
On the end device (Wearable device), we modify the system (Android) behavior, 
so that it can complete the service via connected Access Point (AP), instead of local-hub.
Most importantly, this modification is transparent which means it doesn't affect the user and app developer of wearable devices.
On the service provider (AP) at edge side, we provide the sharable services, which are required by wearable device.
This system not only enables wearable devices can be functional within a wider connection range (Compared with Bluetooth paring), 
but also reduce the long network latency delay caused by original mechanism.
Through the experiments, the results show that we reduce the execution time (Network latency) of services by 2.29x. \\

{\bf Keywords} : 

\end{Abstract_E}
